% ************* Make changes after \begin{document} ***************
%
%  August 07: original template is from 
%  http://www.slac.stanford.edu/econf/editors/eprint-template/instructions.html
%             Modified for CHARM 2007
%
%% ****** Start of file slactemplate.tex ****** %
%%
%%
%%   This file is part of the APS files in the REVTeX 4 distribution.
%%   Version 4.0 of REVTeX, August 2001
%%
%%
%%   Copyright (c) 2001 The American Physical Society.
%%
%%   See the REVTeX 4 README file for restrictions and more information.
%%
%
% This is a template for producing manuscripts for use with REVTEX 4.0
% Copy this file to another name and then work on that file.
% That way, you always have this original template file to use.
%
\documentclass[twocolumn,twoside,slac_two]{revtex4}
\usepackage{graphicx}
\usepackage{fancyhdr}
\pagestyle{fancy}
\fancyhead{} % clear all fields
\fancyhead[C]{\it {
Proceedings of the DPF-2009 Conference, Detroit, MI, July 27-31, 2009
}} \fancyhead[RO,LE]{\thepage}
\fancyfoot{} % clear all fields
\fancyfoot[LE,LO]{}

\renewcommand{\headrulewidth}{0pt}
\renewcommand{\footrulewidth}{0pt}
\renewcommand{\sfdefault}{phv}

\setlength{\textheight}{235mm}
\setlength{\textwidth}{170mm}
\setlength{\topmargin}{1mm}

\bibliographystyle{apsrev}

% ************* Make changes after here  ***************

\begin{document}

%Title of paper
\title{LaTeX Template for DPF-2009 Conference Proceedings}

% Repeat the \author .. \affiliation  etc. as needed
%
% \affiliation command applies to all authors since the last
% \affiliation command. The \affiliation command should follow the
% other information

\author{A.A. Petrov}
\affiliation{Department of Physics and Astronomy, Wayne State University, Detroit, MI 48201, USA}
%
%

\begin{abstract}
This document serves as a template for the proceedings of the DPF-2009 conference.  
Authors should prepare their papers using a copy and follow the guidelines described here.
Please do not modify the page layout or styles. 
\end{abstract}

%\maketitle must follow title, authors, abstract
\maketitle

\thispagestyle{fancy}

% body of paper here - Use proper section commands
% References should be done using the \cite, \ref, and \label commands
% Put \label in argument of \section for cross-referencing
%\section{\label{}}

%%%%%%%%%%%%%%%%%%%%%%%%%%%%%%%%%%
\section{Introduction}
Proceedings of the DPF-2009 Conference
will be made available at the Electronic Conference Proceedings 
Archive\footnote{http://www.slac.stanford.edu/econf/search.html}
under eConf 090726.  
The purpose of this document is to guide the author and to provide a template 
that maintains a uniform style for the proceedings; the next sections
will describe this in detail.

%%%%%%%%%%%%%%%%%%%%%%%%%%%%%%%%%%
\section{Using this Template}
The first step to preparing the paper is to copy the slac\_two.rtx file, 
available at the 
DPF-2009 web site\footnote{http://www.dpf2009.wayne.edu/proceedings.php} 
for download as a template package, into a new directory.
Next copy this template (dpf2009\_template.tex) 
and rename it as dpf2009\_SpeakerName.tex, 
where SpeakerName should be replaced by the author name
(remove spacing between the first and last names as in this example). 
Edit the file to make the necessary modifications and save when finished. 

Because this template has been set up to meet requirements for conference proceedings  
papers, it is important to maintain these established styles.   
Other editorial guidelines are described in the next section.

%%%%%%%%%%%%%%%%%%%%%%%%%%%%%%%%%%
\section{Manuscript}


%%%%%%%%%%%%%%%%%%%%%%%%%%%%%%%%%%
\subsection{Page Limits and Text Layout}
The guidelines for the length of articles are: 6, 8 and 10 pages for talks of length 20, 25 and 
30 minutes, including the time allocated for questions.  Please avoid nestling more than 
three sub-levels in sections; also the first character of section title key words should be
uppercase.

%%%%%%%%%%%%%%%%%%%%%%%%%%%%%%%%%%
\subsection{Acronyms and Abbreviations}
Acronyms should be defined the first time they appear.  
Abbreviations may be used for figures (Fig.), equations (Eq.) 
or if they are commonly used in language (etc., e.g., i.e., v.s.).
However, when starting a sentence, abbreviations should not be used.

%%%%%%%%%%%%%%%%%%%%%%%%%%%%%%%%%%
\subsection{Tables}
Tables may be one (e.g.~Table~\ref{example_table})
or two columns (e.g.~Table~\ref{example_table_2col})
in width with single border lines.
The caption should appear above the table and, 
in the tex file, labels should be used to refer to tables.

% tables should appear as floats within the text
%
% Here is an example of the general form of a table:
% Fill in the caption in the braces of the \caption{} command. Put the label
% that you will use with \ref{} command in the braces of the \label{} command.
% Insert the column specifiers (l, r, c, d, etc.) in the empty braces of the
% \begin{tabular}{} command.
% The ruledtabular environment adds doubled rules to table and sets a
% reasonable default table settings.
% Use the table* environment to get a full-width table in two-column
% Add \usepackage{longtable} and the longtable (or longtable*}
% environment for nicely formatted long tables. Or use the the [H]
% placement option to break a long table (with less control than
% in longtable).
% \begin{table}%[H] add [H] placement to break table across pages
% \caption{\label{}}
% \begin{ruledtabular}
% \begin{tabular}{}
% Lines of table here ending with \\
% \end{tabular}
% \end{ruledtabular}
% \end{table}
% Surround table environment with turnpage environment for landscape
% table
% \begin{turnpage}
% \begin{table}
% \caption{\label{}}
% \begin{ruledtabular}
% \begin{tabular}{}
% \end{tabular}
% \end{ruledtabular}
% \end{table}
% \end{turnpage}

\begin{table}[h]
\begin{center}
\caption{Example of a Table.}
\begin{tabular}{|l|c|c|c|}
\hline \textbf{Margin} & \textbf{Dual} & \textbf{A4 Paper} &
\textbf{US Letter Paper}
\\
\hline Top & 7.6 mm & 37 mm & 19 mm \\
 & (0.3 in) & (1.45 in) & (0.75 in) \\
\hline Bottom & 20 mm & 19 mm & 19 mm \\
 & (0.79 in) & (0.75 in)& (0.75 in) \\
\hline Left & 20 mm & 20 mm & 20 mm \\
 & (0.79 in) & (0.79 in) & (0.79 in) \\
\hline Right & 20 mm & 20 mm & 26 mm \\
 & (0.79 in) & (0.79 in) & (1.0 in) \\
\hline
\end{tabular}
\label{example_table}
\end{center}
\end{table}

\begin{table*}[t]
\begin{center}
\caption{Example of a Full Width Table.}
\begin{tabular}{|l|c|c|c|c|c|c||l|c|c|c|c|c|c|}
\hline \textbf{H-Margin} & \multicolumn{2}{c|}{\textbf{Dual}} & 
\multicolumn{2}{c|}{\textbf{A4}} &
\multicolumn{2}{c||}{\textbf{US Letter}} &
\textbf{V-Margins} & \multicolumn{2}{c|}{\textbf{Dual}} &
\multicolumn{2}{c|}{\textbf{A4}} &
\multicolumn{2}{c|}{\textbf{US Letter}} \\
\hline 
Left & 20 mm & 0.79 in & 20 mm & 0.79 in & 20 mm & 0.79 in &
Top & 7.6 mm & 0.3 in & 37 mm & 1.45 in & 19 mm & 0.75 in \\
\hline 
Right & 20 mm & 0.79 in & 20 mm & 0.79 in & 26 mm & 1.0 in &
Bottom & 20 mm & 0.79 in  & 19 mm & 0.75 in & 19 mm & 0.75 in \\
\hline
\end{tabular}
\label{example_table_2col}
\end{center}
\end{table*}

%%%%%%%%%%%%%%%%%%%%%%%%%%%%%%%%%%
\subsection{Figures}

% figures should be put into the text as floats.
% Use the graphics or graphicx packages (distributed with LaTeX2e)
% and the \includegraphics macro defined in those packages.
% See the LaTeX Graphics Companion by Michel Goosens, Sebastian Rahtz,
% and Frank Mittelbach for instance.
%
% Here is an example of the general form of a figure:
% Fill in the caption in the braces of the \caption{} command. Put the label
% that you will use with \ref{} command in the braces of the \label{} command.
% Use the figure* environment if the figure should span across the
% entire page. There is no need to do explicit centering.

% \begin{figure}
% \includegraphics{}%
% \caption{\label{}}
% \end{figure}

% Surround figure environment with turnpage environment for landscape
% figure
% \begin{turnpage}
% \begin{figure}
% \includegraphics{}%
% \caption{\label{}}
% \end{figure}
% \end{turnpage}

\begin{figure}[h]
\centering
\includegraphics[width=80mm]{figure1.eps}
\caption{Example of a One-column Figure.} \label{example_figure}
\end{figure}

\begin{figure*}[t]
\centering
\includegraphics[width=135mm]{figure2.eps}
\caption{Example of a Full Width Figure.} \label{example_figure_col2}
\end{figure*}

Figures may be one column (e.g.~Figure~\ref{example_figure}) 
or span the paper width (e.g.~Figure~\ref{example_figure_col2}).
Labels should be used to refer to the figures in the tex file.
The figure caption should appear below the figure.  
Color figures are encouraged; however note that some colors, such as 
pale yellow, will not be visible when printed.  
Lettering in figures should be large enough to be visible.
Filenames of figures in the tex file should be numbered
consecutively, e.g. figure1.ps, figure2a.ps, figure2b.ps, figure3.ps, etc.

%%%%%%%%%%%%%%%%%%%%%%%%%%%%%%%%%%
\subsection{Equations}
Equations may be displayed in several ways, as in Eq.~\ref{eq-xy} 

\begin{equation}
x = y.
\label{eq-xy}
\end{equation}

The {\em eqnarray} environment may be used to
split equations into several lines or to align several equations, 
for example in Eq.~\ref{eq-sp}

\begin{eqnarray}
T & = & Im[V_{11} {V_{12}}^* {V_{21}}^* V_{22}]  \nonumber \\
&  & + Im[V_{12} {V_{13}}^* {V_{22}}^* V_{23}]   \nonumber \\
&  & - Im[V_{33} {V_{31}}^* {V_{13}}^* V_{11}].
\label{eq-sp}
\end{eqnarray}

An alternative method is shown in Eq.~\ref{eq-spa} for long sets of
equations where only one referencing equation number is required.  
Again, references should be to labels and not hard-coded equation numbers.

\begin{equation}
\begin{array}{rcl}
\bf{K} & = &  Im[V_{j, \alpha} {V_{j,\alpha + 1}}^*
{V_{j + 1,\alpha }}^* V_{j + 1, \alpha + 1} ] \\
       &   & + Im[V_{k, \alpha + 2} {V_{k,\alpha + 3}}^*
{V_{k + 1,\alpha + 2 }}^* V_{k + 1, \alpha + 3} ]  \\
       &   & + Im[V_{j + 2, \beta} {V_{j + 2,\beta + 1}}^*
{V_{j + 3,\beta }}^* V_{j + 3, \beta + 1} ]  \\
       &   & + Im[V_{k + 2, \beta + 2} {V_{k + 2,\beta + 3}}^*
{V_{k + 3,\beta + 2 }}^* V_{k + 3, \beta + 3}], \\
& & \\
\bf{M} & = &  Im[{V_{j, \alpha}}^* V_{j,\alpha + 1}
V_{j + 1,\alpha } {V_{j + 1, \alpha + 1}}^* ]  \\
       &   & + Im[V_{k, \alpha + 2} {V_{k,\alpha + 3}}^*
{V_{k + 1,\alpha + 2 }}^* V_{k + 1, \alpha + 3} ]  \\
       &   & + Im[{V_{j + 2, \beta}}^* V_{j + 2,\beta + 1}
V_{j + 3,\beta } {V_{j + 3, \beta + 1}}^* ]  \\
       &   & + Im[V_{k + 2, \beta + 2} {V_{k + 2,\beta + 3}}^*
{V_{k + 3,\beta + 2 }}^* V_{k + 3, \beta + 3}].
\\ & &
\end{array}\label{eq-spa}
\end{equation}

%%%%%%%%%%%%%%%%%%%%%%%%%%%%%%%%%%
\subsection{Footnotes} 
Footnotes should only be used in the body of the
paper and not placed after the list of authors, affiliations, or
in the abstract. 

%%%%%%%%%%%%%%%%%%%%%%%%%%%%%%%%%%
\section{Paper Submission}

Authors should submit their papers to the ePrint arXiv 
server\footnote{http://arxiv.org/help} 
after verifying that it is processed correctly by the LaTeX processor.
Please submit the source code, the style files 
(revsymb.sty, revtex4.cls, slac\_two.rtx) 
and any figures; 
these should be self-contained to generate the paper from source.  

It is the author's responsibility to ensure that the papers are 
generated correctly from the source code at the ePrint server. 
After the paper is accepted by the ePrint server, please verify that
the layout in the resulting  PDF file conforms to the guidelines 
described in this document.
Finally, contact the organizers of DPF-2009 and your parallel session conveners 
(see http://www.dpf2009.wayne.edu) with the ePrint number of the paper; 
the deadline to do this is 2~October~2009.

% If you have acknowledgments, this puts in the proper section head.
%\bigskip % extra skip inserted
%%%%%%%%%%%%%%%%%%%%%%%%%%%%%%%%%%
\begin{acknowledgments}
This document is adapted from the instructions provided to the authors
of the proceedings papers at CHARM~07, Ithaca, NY~\cite{charm07},  
and from eConf templates~\cite{templates-ref}.
\end{acknowledgments}

\bigskip % extra skip inserted
% Create the reference section using BibTeX:
%\bibliography{basename of .bib file}
\begin{thebibliography}{9}   % Use for  1-9  references
%\begin{thebibliography}{99} % Use for 10-99 references

\bibitem{charm07}   http://www.lepp.cornell.edu/charm07/

\bibitem{templates-ref} http://www.slac.stanford.edu/econf/editors/eprint-template/instructions.html

\end{thebibliography}


\end{document}

